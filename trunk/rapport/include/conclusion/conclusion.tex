\chapter{Conclusion}

	\section{Avis général}

		Le Taquin a été une approche pratique intéressante concernant le Pattern MVC. L'intérêt de celui-ci apparaît très clairement lors du changement de JPanel où l'on conserve le même modèle (\textit{TaquinGrid}) bien que la manière d'afficher l'information diffère, soit en affichant une image, soit en affichant des chiffres. Le pattern permet également une nette séparation entre les classes modèles qui se retrouvent réutilisables dans un autre contexte, dû au fait qu'elles n'embarquent pas de code spécifique au contrôle des évènements ou à l'affichage de la vue.

		Le seul bémol de ce projet concerne la taille des groupes. En effet, cela a été très difficile de départager les différentes tâches à effectuer en 4. Plusieurs personnes devaient donc travailler ensemble sur un même code car il était impossible de travailler sur 4 tâches différentes en simultané.

	\section{Éléments à améliorer}

		Bien que tous les éléments demandés soient remplis, voici une liste non exhaustive d'améliorations possibles :

		\begin{itemize}
			\item{Affichage d'un compteur de coups}
			\item{Ajout d'un système de score prenant en compte le temps mis pour résoudre la grille ainsi que le nombre de coups}
			\item{Sauvegarde des scores avec le nom d'utilisateur du joueur}
			\item{Ajout d'un mode versus où 2 joueurs seraient en compétition pour finir le plus rapidement possible une même grille}
		\end{itemize}
