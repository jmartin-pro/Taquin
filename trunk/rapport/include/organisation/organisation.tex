\chapter{Organisation du projet}



	\section{Gestion du projet}

		Afin de faciliter la communication et le bon déroulement de la conception de notre application, divers moyens ont été mis en oeuvre.

		\subsection{Gestionnaire de version}

			Tout d'abord, nous pouvons citer l'utilisation d'un gestionnaire de version afin de permettre la centralisation du code et rendre le travail en équipe bien plus efficace. Le choix de celui-ci étant imposé (\textit{Subversion}), il n'est pas nécessaire d'en parler plus longtemps.

		\subsection{Trello}

			Concernant la répartition et le "listing" du travail à effectuer, nous avons fais le choix d'utiliser \href{https://trello.com}{Trello}, une plateforme qui nous permet d'utiliser des tableaux pour plannifier un projet.

			\begin{figure}[H]
				\centering\includegraphics[width=0.75\textwidth, keepaspectratio]{img/trello.png}
				\caption{Notre tableau Trello}
				\label{fig:trello}
			\end{figure}

			Ainsi, comme nous pouvons le constater, les différentes tâches passent par différents états, "\textit{A faire}", "\textit{En cours}", "\textit{A vérifier}", "\textit{Fini et merge}". Enfin, bien que ce ne soit pas visible sur l'image \ref{fig:trello}, il existe un "\textit{Backlog}" sur la droite qui contient les différentes tâches restantes à accomplir. Celles-ci peuvent ensuite être déplacées dans la colonne "\textit{A faire}" au moment où nous jugeons qu'elles peuvent être réalisées.

			Les colonnes "\textit{A verifier}" et "\textit{Fini et merge}" nécessitent quelques précisions, les autres parlant d'elles-même. Pour la première, lorsqu'une tâche est terminée, elle est soumise à évaluation et relecture. Cela permet d'obtenir un avis sur la fonctionnalité et d'éviter d'éventuels bugs par la suite mais aussi de garder une cohérence au travers du code. Raisons pour lesquelles les personnes qui effectuent cette relecture sont souvent les mêmes. Enfin, quand celle-ci est vérifiée et validée, on peut alors la déplacer dans la seconde colonne.

		\subsection{Discord}

			Afin de faciliter la communication au sein du groupe, nous avons utilisé le service de messagerie \href{https://discordapp.com}{Discord} car tous les membres du groupe l'utilisaient déjà de manière personnelle. Celui-ci permet de parler par le biais de "serveurs" gratuits dans lesquels nous pouvons ajouter des salons textuels ou des salons vocaux à volonté. Ainsi, nous avions deux salons de discussion. L'un nommé "\textit{important-taquin}" permet de transmettre des messages importants sur ce qui a été fait, sur des changements importants concernant le projet, etc. L'autre se nommant "\textit{dev-taquin}" était une discution beaucoup plus générale dans laquelle on pouvait demander de l'aide, aider des membres en difficulté, ou même de discuter de certains choix à faire.

			\begin{figure}[H]
				\centering\includegraphics[width=0.75\textwidth, keepaspectratio]{img/discord.png}
				\caption{Notre serveur Discord}
			\end{figure}

	\section{Répartition des fonctionnalités}
