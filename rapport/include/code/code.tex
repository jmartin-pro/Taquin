\chapter{Aspects techniques}

\documentclass{article}
\usepackage[utf8]{inputenc}
\usepackage{algorithm}
\usepackage{algorithmic}

\begin{document}

\section{Algo}
Tout d'abord, voici la class TaquinGrid qui permet de créer et de gérer la grille, que cela soit pour mélanger, effectuer les déplacements, ou encore dire si le joueur à gagner ou pas.
\newline
\\
Cette algo mermet de créer la grille du taquin. Le constructeur de la class TaquinGrid est avant tout appelé permettant d'avoir les \textit{width} et \textit{height} de la grille.
\begin{algorithm}
    \caption{createGrid():void}
    \begin{algorithmic}
        \STATE $this.grid \leftarrow $int $[this.width][this.height]$
        
        \FOR{$y = 0; y < this.height; y++$}
            \FOR{$x = 0; x < this.width; x++$}
                \STATE $this.grid[x][y] \leftarrow x+y*this.width+1$
            \ENDFOR
        \ENDFOR
        
        \STATE $this.grid[this.width-1][this.height-1] \leftarrow -1$
        \STATE $this.posXVide \leftarrow this.width-1$
        \STATE $this.posYVide \leftarrow this.height-1$
    \end{algorithmic}
\end{algorithm}
\newline
La méthode randomizeGrid() permet de mélanger la grille \textit{n} fois.
\\
\begin{algorithm}
    \caption{randomizeGrid(int n):void}
    \begin{algorithmic}
        \STATE $r\leftarrow new$ Randow()
        
        \FOR{int $i = 0; i < n; i++$}
            \STATE $nbrRandom \leftarrow r.nextInt(4)$
            \STATE  $dir \leftarrow null$
            \IF{$nbrRandom == 0$}
                \STATE $dir \leftarrow HAUT$
            \ELSIF{$nbrRandom == 1$}
                \STATE $dir \leftarrow DROITE$
            \ELSIF{$nbrRandom == 1$}
                \STATE $dir \leftarrow BAS$
            \ELSIF{$nbrRandom == 1$}
                \STATE $dir \leftarrow GAUCHE$
            \ENDIF
            
            \IF{$!move(dir)$}
                \STATE $i \leftarrow 1-i$
            \ENDIF
        \ENDFOR
        
        \IF{$finisehd()$}
            \STATE $randomieGrid(int n)$
        \ENDIF
    \end{algorithmic}
\end{algorithm}
\newline
La méthode move() permet d'effectuer des déplacement dans la grille grâce aux mouvement que l'on envoie dans la méthode (variable \textit{direction}).
\begin{algorithm}
    \caption{move(Direction direction):boolean}
    \begin{algorithmic}
        \IF{$direction $ == $HAUT$ \AND $this.posYVide $ == $this.height-1$}
            \RETURN false
        \ELSIF{$direction $== $DROITE$ \AND $this.posXVide $== $0$}
            \RETURN false
        \ELSIF{$direction $== $BAS$ \AND $this.posYVide $== $0$}
            \RETURN false
        \ELSIF{$direction $== $GAUCHE$ \AND $this.posXVide $== $this.width-1$}
            \RETURN false
        \ENDIF
            
        \IF{$direction $ == $HAUT$}
            \STATE $this.grid[posXVide][posYVide] \leftarrow this.grid[posXVide][posYVide+1]$
            \STATE $this.grid[posXVide][posYVide+1] \leftarrow -1$
            \STATE $this.posYVide++$
        \ELSIF{$direction $ == $DROITE$}
            \STATE $this.grid[posXVide][posYVide] \leftarrow this.grid[posXVide-1][posYVide]$
            \STATE $this.grid[posXVide-1][posYVide] \leftarrow -1$
            \STATE $this.posYVide--$
        \ELSIF{$direction $ == $BAS$}
            \STATE $this.grid[posXVide][posYVide] \leftarrow this.grid[posXVide][posYVide-1]$
            \STATE $this.grid[posXVide][posYVide-1] \leftarrow -1$
            \STATE $this.posYVide--$
        \ELSIF{$direction $ == $GAUCHE$}
            \STATE $this.grid[posXVide][posYVide] \leftarrow this.grid[posXVide+1][posYVide]$
            \STATE $this.grid[posXVide+1][posYVide] \leftarrow -1$
            \STATE $this.posYVide++$
        \ENDIF
        \RETURN true
    \end{algorithmic}
\end{algorithm}

\end{document}
